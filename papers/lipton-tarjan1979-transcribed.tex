\documentclass[12pt]{article}

% ---- Packages ----
\usepackage{amsmath, amssymb, amsthm}
\usepackage{geometry}
\usepackage{graphicx}
\usepackage{hyperref}
\usepackage{enumitem}
\usepackage{titlesec}
\usepackage{mathtools}

\geometry{letterpaper, margin=1.2in}

% ---- Theorem environments ----
\newtheorem{theorem}{Theorem}
\newtheorem{lemma}{Lemma}
\newtheorem{corollary}{Corollary}

\theoremstyle{remark}
\newtheorem*{remark}{Remark}

% ---- Title formatting ----
\titleformat{\section}{\normalfont\bfseries\large}{\thesection.}{1em}{}
\titleformat{\subsection}{\normalfont\bfseries}{\thesubsection}{1em}{}

% ---- Custom commands ----
\newcommand{\bigO}{\mathcal{O}}
\newcommand{\sqrtn}{\sqrt{n}}

% ====================================================================
\begin{document}

\begin{center}
  {\small SIAM J.\ APPL.\ MATH.\ Vol.\ 36, No.\ 2, April 1979}\\[2pt]
  {\small \copyright\ 1979 Society for Industrial and Applied Mathematics}\\[2pt]
  {\small 0036-1399/79/3602-0001 \$01.00/0}
  \vspace{1.2em}

  {\Large\bfseries A SEPARATOR THEOREM FOR PLANAR GRAPHS\footnote{Received by
  the editors August 10, 1977.}}

  \vspace{0.8em}
  {\large RICHARD J. LIPTON\footnote{Computer Science Department, Yale
  University, New Haven, Connecticut 06520. This research was supported in part
  by the U.S.\ Army Research Office under Grant DAAG 29-76-G-0338 and The
  National Science Foundation under Grant MCS 78-81486.}\; AND ROBERT ENDRE
  TARJAN\footnote{Computer Science Department, Stanford University, Stanford,
  California 94305. This research was supported in part by National Science
  Foundation under Grant MCS-75-22870 and in part by the Office of Naval
  Research under Contract N00014-76-C-0688.}}
\end{center}

\begin{abstract}
Let $G$ be any $n$-vertex planar graph. We prove that the vertices of $G$ can be
partitioned into three sets $A$, $B$, $C$ such that no edge joins a vertex in $A$
with a vertex in $B$, neither $A$ nor $B$ contains more than $2n/3$ vertices, and
$C$ contains no more than $2\sqrt{2}\sqrt{n}$ vertices. We exhibit an algorithm
which finds such a partition $A$, $B$, $C$ in $\bigO(n)$ time.
\end{abstract}

% ====================================================================
\section{Introduction}

A useful method for solving many kinds of combinatorial problems is
``divide-and-conquer''~\cite{aho74}. In this method the problem of interest is
divided into two or more smaller problems. The subproblems are solved by applying
the method recursively, and the subproblem solutions are combined to give the
solution to the original problem. Three things are necessary for the success and
efficiency of divide-and-conquer: (i) the subproblems must be of the same type as
the original and independent of each other (in a suitable sense); (ii) the cost of
solving the original problem given the solutions to the subproblems must be small;
and (iii) the subproblems must be significantly smaller than the original. One way to
guarantee that the subproblems are small is to make them all roughly the same
size~\cite{aho74}.

We wish to study general conditions under which the divide-and-conquer approach
is useful. Consider problems which are defined on graphs. Let $S$ be a class of
graphs closed under the subgraph relation (i.e., if $G_1 \in S$ and $G_2$ is a
subgraph of $G_1$, then $G_2 \in S$). An \emph{$f(n)$-separator theorem} for $S$ is
a theorem of the following form:

\begin{quote}
There exist constants $\alpha < 1$, $\beta > 0$ such that if $G$ is any $n$-vertex
graph in $S$, the vertices of $G$ can be partitioned into three sets $A$, $B$, $C$
such that no edge joins a vertex in $A$ with a vertex in $B$, neither $A$ nor $B$
contains more than $\alpha n$ vertices, and $C$ contains no more than $\beta f(n)$
vertices.
\end{quote}

If such a theorem holds for the class of graphs $S$, and if the appropriate vertex
partitions $A$, $B$, $C$ can be found fast, then a number of problems defined on
graphs in $S$ can be solved efficiently using divide-and-conquer. For a given graph
$G$ in $S$, the sets $A$ and $B$ define the subproblems. The cost of combining the
subproblem solutions is a function of the size of $C$ (and thus of $f(n)$).

Previously known separator theorems include the following:

\begin{enumerate}[label=(\Alph*)]
\item Any $n$-vertex binary tree can be separated into two subtrees, each with no
more than $2n/3$ vertices, by removing a single edge. For an application of this
theorem, see~\cite{lewis65}.

\item Any $n$-vertex tree can be divided into two parts, each with no more than
$2n/3$ vertices, by removing a single vertex.

\item A \emph{grid graph} is any subgraph of the infinite two-dimensional square
grid illustrated in Fig.~1. A $\sqrt{n}$-separator theorem holds for the class of
grid graphs. For an application, see~\cite{george73}.

\item A \emph{one-tape Turing machine graph}~\cite{paul77} is a graph representing
the computation of a one-tape Turing machine. A $\sqrt{n}$-separator theorem holds
for such graphs. For an application, see~\cite{paterson72}.
\end{enumerate}

One might conjecture that the class of all suitably sparse graphs has an
$f(n)$-separator theorem for some $f(n)=o(n)$. However, the following result of
Erd\H{o}s, Graham and Szemer\'{e}di~\cite{erdos75} shows that this is not the case.

\begin{theorem}[\cite{erdos75}]
\label{thm:erdos}
For every $\varepsilon > 0$ there is a positive constant $c = c(\varepsilon)$ such
that almost all\footnote{By ``almost all'' we mean that the fraction of graphs
possessing the property tends with increasing $n$ to one.} graphs $G$ with
$n = (2+\varepsilon)k$ vertices and $ck$ edges have the property that after the
omission of any $k$ vertices, a connected component of at least $k$ vertices
remains.
\end{theorem}

Although sparsity by itself is not enough to give a useful separator theorem,
planarity is. In \S2 of this paper we prove that a $\sqrt{n}$-separator theorem holds
for all planar graphs. In \S3 we provide a linear-time algorithm for finding a vertex
partition satisfying the theorem. This algorithm and the divide-and-conquer approach
combine to give efficient algorithms for a wide range of problems on planar graphs.
Section~4 mentions some of these applications, which we shall discuss more fully in a
subsequent paper.

% ====================================================================
\section{Separator theorems}

To prove our results we need to use three facts about planarity.

\begin{theorem}[Jordan curve theorem~\cite{hall55}]
\label{thm:jordan}
Let $C$ be any closed curve in the plane. Removal of $C$ divides the plane into
exactly two connected regions, the ``inside'' and the ``outside'' of $C$.
\end{theorem}

\begin{theorem}[\cite{harary69}]
\label{thm:edges}
Any $n$-vertex planar graph with $n \geq 3$ contains no more than $3n-6$ edges.
\end{theorem}

\begin{theorem}[Kuratowski's theorem~\cite{kuratowski30}]
\label{thm:kuratowski}
A graph is planar if and only if it contains neither a complete graph on five
vertices (Fig.~2(a)) nor a complete bipartite graph on two sets of three vertices
(Fig.~2(b)) as a generalized subgraph.
\end{theorem}

From Kuratowski's theorem we can easily obtain the following lemma and its corollary.

\begin{lemma}
\label{lem:shrink-edge}
Let $G$ be any planar graph. Shrinking any edge of $G$ to a single vertex preserves
planarity.
\end{lemma}

\begin{proof}
Let $G^*$ be the shrunken graph, let $(x_1, x_2)$ be the edge shrunk, and let $x$
be the vertex corresponding to $x_1$ and $x_2$ in $G^*$. If $G^*$ is not planar
then $G^*$ contains a Kuratowski graph as a generalized subgraph. But this subgraph
corresponds to a Kuratowski graph which is a generalized subgraph of $G$. Figure~3
illustrates the possibilities.
\end{proof}

\begin{corollary}
\label{cor:shrink-subgraph}
Let $G$ be any planar graph. Shrinking any connected subgraph of $G$ to a single
vertex preserves planarity.
\end{corollary}

\begin{proof}
The proof is immediate from Lemma~\ref{lem:shrink-edge} by induction on the number
of vertices in the subgraph to be shrunk.
\end{proof}

In some applications it is useful to have a result more general than the kind of
separator theorem described in the Introduction. We shall therefore consider planar
graphs which have nonnegative costs on the vertices. We shall prove that any such
graph can be separated into two parts, each with cost no more than two-thirds of the
total cost, by removing $\bigO(\sqrt{n})$ vertices. The desired separator theorem is
the special case of equal-cost vertices.

\begin{lemma}
\label{lem:radius}
Let $G$ be any planar graph with nonnegative vertex costs summing to no more than
one. Suppose $G$ has a spanning tree of radius $r$. Then the vertices of $G$ can be
partitioned into three sets $A$, $B$, $C$, such that no edge joins a vertex in $A$
with a vertex in $B$, neither $A$ nor $B$ has total cost exceeding $2/3$, and $C$
contains no more than $2r+1$ vertices, one the root of the tree.
\end{lemma}

\begin{proof}
Assume no vertex has cost exceeding $1/3$; otherwise the lemma is true. Embed $G$
in the plane. Make each face a triangle by adding a suitable number of additional
edges. Any nontree edge (including each of the added edges) forms a simple cycle
with some of the tree edges. This cycle is of length at most $2r+1$ if it contains the
root of the tree, at most $2r-1$ otherwise. The cycle divides the plane (and the
graph) into two parts, the inside and the outside of the cycle. We claim that at least
one such cycle separates the graph so that neither the inside nor the outside contains
vertices whose total cost exceeds $2/3$. This proves the lemma.

\textit{Proof of claim.} Let $(x, z)$ be the nontree edge whose cycle minimizes the
maximum cost either inside or outside the cycle. Break ties by choosing the nontree
edge whose cycle has the smallest number of faces on the same side as the maximum
cost. If ties remain, choose arbitrarily.

Suppose without loss of generality that the graph is embedded so that the cost
inside the $(x, z)$ cycle is at least as great as the cost outside the cycle. If the
vertices inside the cycle have total cost not exceeding $2/3$, the claim is true.
Suppose the vertices inside the cycle have total cost exceeding $2/3$. We show by
case analysis that this contradicts the choice of $(x, z)$. Consider the face which
has $(x, z)$ as a boundary edge and lies inside the cycle. This face is a triangle;
let $y$ be its third vertex. The properties of $(x, y)$ and $(y, z)$ determine which
of the following cases applies. Figure~4 illustrates the cases.

\begin{enumerate}[label=\arabic*)]
\item Both $(x, y)$ and $(y, z)$ lie on the cycle. Then the face $(x, y, z)$ is the
cycle, which is impossible since vertices lie inside the cycle.

\item One of $(x, y)$ and $(y, z)$ (say $(x, y)$) lies on the cycle. Then $(y, z)$
is a nontree edge defining a cycle which contains within it the same vertices as the
original cycle but one less face. This contradicts the choice of $(x, z)$.

\item Neither $(x, y)$ nor $(y, z)$ lies on the cycle.
  \begin{enumerate}[label=\alph*)]
  \item Both $(x, y)$ and $(y, z)$ are tree edges. This is impossible since the tree
  itself contains no cycles.

  \item One of $(x, y)$ and $(y, z)$ (say $(x, y)$) is a tree edge. Then $(y, z)$ is
  a nontree edge defining a cycle which contains one less vertex (namely $y$) within
  it than the original cycle. The inside of the $(y, z)$ cycle contains no more cost
  and one less face than the inside of the $(x, z)$ cycle. Thus if the cost inside the
  $(y, z)$ cycle is greater than the cost outside the cycle, $(y, z)$ would have been
  chosen in place of $(x, z)$.

  On the other hand, suppose the cost inside the $(y, z)$ cycle is no greater than
  the cost outside. The cost outside the $(y, z)$ cycle is equal to the cost outside
  the $(x, z)$ cycle plus the cost of $y$. Since both the cost outside the $(x, z)$
  cycle and the cost of $y$ are less than $1/3$, the cost outside the $(y, z)$ cycle
  is less than $2/3$, and $(y, z)$ would have been chosen in place of $(x, z)$.

  \item Neither $(x, y)$ nor $(y, z)$ is a tree edge. Then each of $(x, y)$ and
  $(y, z)$ defines a cycle, and every vertex inside the $(x, z)$ cycle is either
  inside the $(x, y)$ cycle, inside the $(y, z)$ cycle, or on the boundary of both.
  Of the $(x, y)$ and $(y, z)$ cycles, choose the one (say $(x, y)$) which has inside
  it more total cost. The $(x, y)$ cycle has no more cost and strictly fewer faces
  inside it than the $(x, z)$ cycle. Thus if the cost inside the $(x, y)$ cycle is
  greater than the cost outside, $(x, y)$ would have been chosen in place of $(x, z)$.

  On the other hand, suppose the cost inside the $(x, y)$ cycle is no greater than
  the cost outside. Since the inside of the $(x, z)$ cycle has cost exceeding $2/3$,
  the $(x, y)$ cycle and its inside together have cost exceeding $1/3$, and the
  outside of the $(x, y)$ cycle has cost less than $2/3$. Thus $(x, y)$ would have
  been chosen in place of $(x, z)$.
  \end{enumerate}
\end{enumerate}

Thus all cases are impossible, and the $(x, z)$ cycle satisfies the claim.
\end{proof}

\begin{lemma}
\label{lem:levels}
Let $G$ be any $n$-vertex connected planar graph having nonnegative vertex costs
summing to no more than one. Suppose that the vertices of $G$ are partitioned into
levels according to their distance from some vertex $v$, and that $L(l)$ denotes the
number of vertices on level $l$. If $r$ is the maximum distance of any vertex from
$v$, let $r+1$ be an additional level containing no vertices. Given any two levels
$l_1$ and $l_2$ such that levels $0$ through $l_1 - 1$ have total cost not exceeding
$2/3$ and levels $l_2 + 1$ through $r+1$ have total cost not exceeding $2/3$, it is
possible to find a partition $A$, $B$, $C$ of the vertices of $G$ such that no edge
joins a vertex in $A$ with a vertex in $B$, neither $A$ nor $B$ has total cost
exceeding $2/3$, and $C$ contains no more than
$L(l_1) + L(l_2) + \max\{0, \, 2(l_2 - l_1 - 1)\}$ vertices.
\end{lemma}

\begin{proof}
If $l_1 \geq l_2$, let $A$ be all vertices on levels $0$ through $l_1 - 1$, $B$ all
vertices on levels $l_1 + 1$ through $r$, and $C$ all vertices on level $l_1$. Then
the lemma is true. Thus suppose $l_1 < l_2$. Delete the vertices in levels $l_1$ and
$l_2$ from $G$. This separates the remaining vertices of $G$ into three parts (all of
which may be empty): vertices on levels $0$ through $l_1 - 1$, vertices on levels
$l_1 + 1$ through $l_2 - 1$, and vertices on levels $l_2 + 1$ and above. The only
part which can have cost exceeding $2/3$ is the middle part.

If the middle part does not have cost exceeding $2/3$, let $A$ be the most costly
part of the three, let $B$ be the remaining two parts, and let $C$ be the set of
vertices on levels $l_1$ and $l_2$. Then the lemma is true.

Suppose the middle part has cost exceeding $2/3$. Delete all vertices on levels
$l_2$ and above and shrink all vertices on levels $l_1$ and below to a single vertex
of cost zero. These operations preserve planarity by Corollary~\ref{cor:shrink-subgraph}.
The new graph has a spanning tree of radius $l_2 - l_1 - 1$ whose root corresponds
to vertices on levels $l_1$ and below in the original graph.

Apply Lemma~\ref{lem:radius} to the new graph. Let $A^*$, $B^*$, $C^*$ be the
resulting vertex partition. Let $A$ be the set among $A^*$ and $B^*$ having greater
cost, let $C$ consist of the vertices on levels $l_1$ and $l_2$ in the original graph
plus the vertices in $C^*$ minus the root of the tree, and let $B$ contain the
remaining vertices in $G$. By Lemma~\ref{lem:radius}, $A$ has total cost not
exceeding $2/3$. But $A \cup C^*$ has total cost at least $1/3$, so $B$ also has
total cost not exceeding $2/3$. Furthermore $C$ contains no more than
$L(l_1) + L(l_2) + 2(l_2 - l_1 - 1)$ vertices. Thus the lemma is true.
\end{proof}

\begin{theorem}
\label{thm:main}
Let $G$ be any $n$-vertex planar graph having nonnegative vertex costs summing to
no more than one. Then the vertices of $G$ can be partitioned into three sets $A$,
$B$, $C$ such that no edge joins a vertex in $A$ with a vertex in $B$, neither $A$
nor $B$ has total cost exceeding $2/3$, and $C$ contains no more than
$2\sqrt{2\sqrt{n}}$ vertices.
\end{theorem}

\begin{proof}
Assume $G$ is connected. Partition the vertices into levels according to their
distance from some vertex $v$. Let $L(l)$ be the number of vertices on level $l$. If
$r$ is the maximum distance of any vertex from $v$, define additional levels $-1$
and $r+1$ containing no vertices.

Let $l_1$ be the level such that the sum of costs in levels $0$ through $l_1 - 1$ is
less than $1/2$, but the sum of costs in levels $0$ through $l_1$ is at least $1/2$.
(If no such $l_1$ exists, the total cost of all vertices is less than $1/2$, and
$B = C = \varnothing$ satisfies the theorem.) Let $k$ be the number of vertices on
levels $0$ through $l_1$. Find a level $l_0$ such that $l_0 \leq l_1$ and
$|L(l_0)| + 2(l_1 - l_0) \leq 2\sqrt{k}$. Find a level $l_2$ such that
$l_1 + 1 \leq l_2$ and $|L(l_2)| + 2(l_2 - l_1 - 1) \leq 2\sqrt{n-k}$.
If two such levels exist, then by Lemma~\ref{lem:levels} the vertices of $G$ can be
partitioned into three sets $A$, $B$, $C$ such that no edge joins a vertex in $A$
with a vertex in $B$, neither $A$ nor $C$ has cost exceeding $2/3$, and $C$ contains
no more than $2(\sqrt{k} + \sqrt{n-k})$ vertices. But
$2(\sqrt{k} + \sqrt{n-k}) \leq 2(\sqrt{n/2} + \sqrt{n/2}) = 2\sqrt{2n}$. Thus the
theorem holds if suitable levels $l_0$ and $l_2$ exist.

Suppose a suitable level $l_0$ does not exist. Then, for $i \leq l_1$,
$L(i) \geq 2\sqrt{k} - 2(l_1 - i)$. Since $L(0) = 1$, this means
$1 \geq 2\sqrt{k} - 2l_1$, and $l_1 + 1/2 \geq \sqrt{k}$. Thus
$l_1 = \lfloor l_1 + 1/2 \rfloor \geq \lfloor \sqrt{k} \rfloor$, and
\[
k = \sum_{i=0}^{l_1} L(i) \;\geq\; \sum_{i=l_1 - \lfloor\sqrt{k}\rfloor}^{l_1}
\bigl(2\sqrt{k} - 2(l_1-i)\bigr) \;\geq\;
(4\sqrt{k} - 2\lfloor\sqrt{k}\rfloor)(\lfloor\sqrt{k}\rfloor + 1)/2
\;\geq\; \sqrt{k}(\lfloor\sqrt{k}\rfloor + 1) > k.
\]
This is a contradiction. A similar contradiction arises if a suitable level $l_2$
does not exist. This completes the proof for connected graphs.

Now suppose $G$ is not connected. Let $G_1, G_2, \ldots, G_k$ be the connected
components of $G$, with vertex sets $V_1, V_2, \ldots, V_k$, respectively. If no
connected component has total vertex cost exceeding $1/3$, let $i$ be the minimum
index such that the total cost of $V_1 \cup V_2 \cup \cdots \cup V_i$ exceeds $1/3$.
Let $A = V_1 \cup V_2 \cup \cdots \cup V_i$, let
$B = V_{i+1} \cup V_{i+2} \cup \cdots \cup V_k$, and let $C = \varnothing$. Since
$i$ is minimum and the cost of $V_i$ does not exceed $1/3$, the cost of $A$ does not
exceed $2/3$. Thus the theorem is true.

If some connected component (say $G_i$) has total vertex cost between $1/3$ and
$2/3$, let $A = V_i$,
$B = V_1 \cup \cdots \cup V_{i-1} \cup V_{i+1} \cup \cdots \cup V_k$, and
$C = \varnothing$. Then the theorem is true.

Finally, if some connected component (say $G_i$) has total vertex cost exceeding
$2/3$, apply the above argument to $G_i$. Let $A^*$, $B^*$, $C^*$ be the resulting
partition. Let $A$ be the set among $A^*$ and $B^*$ with greater cost, let $C = C^*$,
and let $B$ be the remaining vertices of $G$. Then $A$ and $B$ have cost not
exceeding $2/3$ and the theorem is true.

This proves the theorem for all planar graphs. In all cases the separator $C$ is
either empty or contained in only one connected component of $G$.
\end{proof}

\begin{corollary}[$\sqrt{n}$-separator theorem]
\label{cor:sqrt-sep}
Let $G$ be any $n$-vertex planar graph. The vertices of $G$ can be partitioned into
three sets $A$, $B$, $C$ such that no edge joins a vertex in $A$ with a vertex in $B$,
neither $A$ nor $B$ contains more than $2n/3$ vertices, and $C$ contains no more
than $2\sqrt{2n}$ vertices.
\end{corollary}

\begin{proof}
Assign to each vertex of $G$ a cost of $1/n$. The corollary follows from
Theorem~\ref{thm:main}.
\end{proof}

It is natural to ask whether the constant factor of $2/3$ in
Theorem~\ref{thm:main} can be reduced to $1/2$ if the constant factor of $2\sqrt{2}$
is allowed to increase. The answer is yes.

\begin{corollary}
\label{cor:half-sep}
Let $G$ be any $n$-vertex planar graph having nonnegative vertex costs summing to
no more than one. Then the vertices of $G$ can be partitioned into three sets $A$,
$B$, $C$ such that no edge joins a vertex in $A$ with a vertex in $B$, neither $A$
nor $B$ has total cost exceeding $1/2$, and $C$ contains no more than
$2\sqrt{2n}/(1 - \sqrt{2/3})$ vertices.
\end{corollary}

\begin{proof}
Let $G = (V, E)$ be an $n$-vertex planar graph. We shall define sequences of sets
$(A_i)$, $(B_i)$, $(C_i)$, $(D_i)$ such that:
\begin{enumerate}[label=(\roman*)]
\item $A_i$, $B_i$, $C_i$, $D_i$ partition $V$,
\item no edge joins $A_i$ with $B_i$, $A_i$ with $D_i$, or $B_i$ with $D_i$,
\item the cost of $A_i$ is no greater than the cost of $B_i$ and the cost of $B_i$
      is no greater than the cost of $A_i \cup C_i \cup D_i$,
\item $|D_i| \leq 2|D_{i-1}|/3$.
\end{enumerate}

Let $A_0 = B_0 = C_0 = \varnothing$, $D_0 = V$. Then (i)--(iv) hold. If $A_{i-1}$,
$B_{i-1}$, $C_{i-1}$, $D_{i-1}$ have been defined and $D_{i-1} \neq \varnothing$,
let $G^*$ be the subgraph of $G$ induced by the vertex set $D_{i-1}$. Let $A^*$,
$B^*$, $C^*$ be a vertex partition satisfying Corollary~\ref{cor:sqrt-sep} on $G^*$.
Without loss of generality, suppose $A^*$ has no more cost than $B^*$. Let $A_i$ be
the set among $A_{i-1} \cup A^*$, $B_{i-1}$ with less cost, let $B_i$ be the set
among $A_{i-1} \cup A^*$, $B_{i-1}$ with greater cost, let $C_i = C_{i-1} \cup C^*$,
and let $D_i = B^*$. Then (i), (ii), (iii), and (iv) hold for $A_i$, $B_i$, $C_i$,
$D_i$.

Let $k$ be the largest index for which $A_k$, $B_k$, $C_k$, $D_k$ are defined.
Then $D_k = \varnothing$. Let $A = A_k$, $B = B_k$, $C = C_k$. By (i), $A$, $B$, $C$
partition $V$. By (ii), no edge joins a vertex in $A$ with a vertex in $B$. By (iii),
neither $A$ nor $B$ has cost exceeding $1/2$. By (iv), the total number of vertices
in $C$ is bounded by
\[
\sum_{i=0}^{\infty} 2\sqrt{2n}(2/3)^{i/2} = \frac{2\sqrt{2n}}{1 - \sqrt{2/3}}. \qedhere
\]
\end{proof}

Another natural question is whether graphs which are ``almost'' planar have a
$\sqrt{n}$-separator theorem. The finite element method of numerical analysis gives
rise to one interesting class of almost-planar graphs. We shall extend
Theorem~\ref{thm:main} to apply to such graphs.

A \emph{finite element graph} is any graph formed from a planar embedding of a
planar graph by adding all possible diagonals to each face. (The finite element graph
has a clique corresponding to each face of the embedded planar graph.) The embedded
planar graph is called the \emph{skeleton} of the finite element graph and each of
its faces is an \emph{element} of the finite element graph.

\begin{theorem}
\label{thm:finite-element}
Let $G$ be an $n$-vertex finite element graph with nonnegative vertex costs summing
to no more than one. Suppose no element of $G$ has more than $k$ boundary vertices.
Then the vertices of $G$ can be partitioned into three sets $A$, $B$, $C$ such that
no edge joins a vertex in $A$ with a vertex in $B$, neither $A$ nor $B$ has total
cost exceeding $2/3$, and $C$ contains no more than $4\lfloor k/2 \rfloor\sqrt{n}$
vertices.
\end{theorem}

\begin{proof}
Let $G^*$ be the skeleton of $G$. Form $G^{**}$ from $G^*$ by inserting one new
vertex into each face of $G^*$ containing four or more vertices and connecting the
new vertex to each vertex on the boundary of the face. Then $G^{**}$ is planar.
Apply Theorem~\ref{thm:main} to $G^{**}$. Let $A^{**}$, $B^{**}$, $C^{**}$ be the
resulting vertex partition. This partition satisfies the theorem except that certain
edges in $G$ but not in $G^{**}$ may join $A^{**}$ and $B^{**}$. These edges are
diagonals of certain faces of $G^*$; call these \emph{bad faces}. Each bad face must
contain one of the new vertices added to $G^*$ to form $G^{**}$, and this vertex
must be in $C^{**}$.

Form $C$ from $C^{**}$ by deleting all new vertices and adding to $C^{**}$, for
each bad face, either the set of vertices in $A^{**}$ on the boundary of the bad
face, or the set of vertices in $B^{**}$ on the boundary of the bad face, whichever
is smaller. Let $A$ be the remaining old vertices in $A^{**}$ and let $B$ be the
remaining old vertices in $B^{**}$. Then no edge in $G$ joins $A$ and $B$, neither
$A$ nor $B$ contains more than $2n/3$ vertices, and $C$ contains no more than
$2\sqrt{2}\lfloor k/2\rfloor\sqrt{n} + a$ vertices, where $a$ is the number of faces
of $G^*$ containing four or more vertices. By use of Euler's theorem, it is not hard
to show that the number of faces of $G^*$ containing four or more vertices is at most
$n - 2$. Thus $|C| \leq \lfloor k/2\rfloor\sqrt{n}$, and the theorem is true.
\end{proof}

\begin{corollary}
\label{cor:finite-element}
Let $G$ be any $n$-vertex finite element graph. Suppose no element of $G$ has more
than $k$ boundary vertices. The vertices of $G$ can be partitioned into three sets
$A$, $B$, $C$ such that no edge joins a vertex in $A$ with a vertex in $B$, neither
$A$ nor $B$ contains more than $2n/3$ vertices and $C$ contains no more than
$4\lfloor k/2\rfloor\sqrt{n}$ vertices.
\end{corollary}

The last result of this section shows that Theorem~\ref{thm:main} and its
corollaries are tight to within a constant factor; that is, if $f(n) = o(\sqrt{n})$,
no $f(n)$-separator theorem holds for planar graphs.

\begin{theorem}
\label{thm:tight}
For any $k$, let $G = (V, E)$ be a $k \times k$ square grid graph (a $k \times k$
square section of the infinite grid graph in Fig.~1). Let $A$ be any subset of $V$
such that $\alpha n \leq |A| \leq n/2$, where $n = k^2$ and $\alpha$ is a positive
constant less than $1/2$. Then the number of vertices in $V \setminus A$ adjacent to
some vertex in $A$ is at least $k \cdot \min\{1/2, \sqrt{\alpha}\}$.
\end{theorem}

\begin{proof}
Without loss of generality, suppose that the number $r$ of rows of $G$ which contain
vertices in $A$ is no less than the number $c$ of columns of $G$ which contain
vertices in $A$. Then $\alpha n \leq |A| \leq rc \leq r^2$ and $r \geq \sqrt{\alpha}k$.

If $r^*$ is the number of rows of $G$ which contain only vertices in $A$, then
$kr^* \leq |A| \leq n/2$, and $r^* \leq k/2$. Let $S = \{x \in V : x \text{ is
adjacent to a vertex of } A\}$. If $r^* = 0$, then $|S| \geq r \geq \sqrt{\alpha}k$.
If $r^* \neq 0$, then $r = k$ and $|S| \geq r - r^* = k - r^* \geq k/2$.
\end{proof}

It is an open problem to determine the smallest constant factor which can replace
$2\sqrt{2}$ in Theorem~\ref{thm:main}.

% ====================================================================
\section{An algorithm for finding a good partition}

The proof of Theorem~\ref{thm:main} leads to an algorithm for finding a vertex
partition satisfying the theorem. To make this algorithm efficient, we need a good
representation of a planar embedding of a graph. For this purpose we use a list
structure whose elements correspond to the edges of the graph. Stored with each edge
are its endpoints and four pointers, designating the edges immediately clockwise and
counter-clockwise around each of the endpoints of the edge. Stored with each vertex
is some incident edge. Figure~5 gives an example of such a data structure.

\bigskip
\noindent\textbf{Partitioning Algorithm.}

\medskip
\noindent\textit{Step 1.} Find a planar embedding of $G$ and construct a
representation for it of the kind described above.

\begin{center}
\textit{Time:} $\bigO(n)$, using the algorithm of \cite{hopcroft74}.
\end{center}

\noindent\textit{Step 2.} Find the connected components of $G$ and determine the cost
of each one. If none has cost exceeding $2/3$, construct the partition as described
in the proof of Theorem~\ref{thm:main}. If some component has cost exceeding $2/3$,
go to Step~3.

\begin{center}
\textit{Time:} $\bigO(n)$ \cite{hopcroft73}.
\end{center}

\noindent\textit{Step 3.} Find a breadth-first spanning tree of the most costly
component. Compute the level of each vertex and the number of vertices $L(l)$ in
each level $l$.

\begin{center}
\textit{Time:} $\bigO(n)$.
\end{center}

\noindent\textit{Step 4.} Find the level $l_1$ such that the total cost of levels $0$
through $l_1 - 1$ does not exceed $1/2$, but the total cost of levels $0$ through
$l_1$ does exceed $1/2$. Let $k$ be the number of vertices in levels $0$ through
$l_1$.

\begin{center}
\textit{Time:} $\bigO(n)$.
\end{center}

\noindent\textit{Step 5.} Find the highest level $l_0 \leq l_1$ such that
$L(l_0) + 2(l_1 - l_0) \leq 2\sqrt{k}$. Find the lowest level $l_2 \geq l_1 + 1$
such that $L(l_2) + 2(l_2 - l_1 - 1) \leq 2\sqrt{n-k}$.

\begin{center}
\textit{Time:} $\bigO(n)$.
\end{center}

\noindent\textit{Step 6.} Delete all vertices on level $l_2$ and above. Construct a
new vertex $x$ to represent all vertices on levels $0$ through $l_0$. Construct a
Boolean table with one entry per vertex. Initialize to \textbf{true} the entry for
each vertex on levels $0$ through $l_0$ and initialize to \textbf{false} the entry
for each vertex on levels $l_0 + 1$ through $l_2 - 1$. The vertices on levels $0$
through $l_0$ correspond to a subtree of the breadth-first spanning tree generated in
Step~3. Scan the edges incident to this tree clockwise around the tree. When scanning
an edge $(v, w)$ with $v$ in the tree, check the table entry for $w$. If it is
\textbf{true}, delete edge $(v, w)$. If it is \textbf{false}, change it to
\textbf{true}, construct an edge $(x, w)$, and delete edge $(v, w)$. The result of
this step is a planar representation of the shrunken graph to which
Lemma~\ref{lem:radius} is to be applied. See Fig.~6.

\begin{center}
\textit{Time:} $\bigO(n)$.
\end{center}

\noindent\textit{Step 7.} Construct a breadth-first spanning tree rooted at $x$ in
the new graph. (This can be done by modifying the breadth-first spanning tree
constructed in Step~3.) Record, for each vertex $v$, the parent of $v$ in the tree,
and the total cost of all descendants of $v$ including $v$ itself. Make all faces of
the new graph into triangles by scanning the boundary of each face and adding
(nontree) edges as necessary.

\begin{center}
\textit{Time:} $\bigO(n)$.
\end{center}

\noindent\textit{Step 8.} Choose any nontree edge $(v_1, w_1)$. Locate the
corresponding cycle by following parent pointers from $v_1$ and $w_1$. Compute the
cost on each side of this cycle by scanning the tree edges incident on either side of
the cycle and summing their associated costs. If $(v, w)$ is a tree edge with $v$ on
the cycle and $w$ not on the cycle, the cost associated with $(v, w)$ is the
descendant cost of $w$ if $v$ is the parent of $w$, and the cost of all vertices
minus the descendant cost of $v$ if $w$ is the parent of $v$. Determine which side
of the cycle has greater cost and call it the ``inside''. See Fig.~7.

\begin{center}
\textit{Time:} $\bigO(n)$.
\end{center}

\noindent\textit{Step 9.} Let $(v_i, w_i)$ be the nontree edge whose cycle is the
current candidate to complete the separator. If the cost inside the cycle exceeds
$2/3$, find a better cycle by the following method.

Locate the triangle $(v_i, y, w_i)$ which has $(v_i, w_i)$ as a boundary edge and
lies inside the $(v_i, w_i)$ cycle. If either $(v_i, y)$ or $(y, w_i)$ is a tree
edge, let $(v_{i+1}, w_{i+1})$ be the nontree edge among $(v_i, y)$ and $(y, w_i)$.
Compute the cost inside the $(v_{i+1}, w_{i+1})$ cycle from the cost inside the
$(v_i, w_i)$ cycle and the cost of $v_i$, $y$, and $w_i$. See Fig.~4.

If neither $(v_i, y)$ nor $(y, w_i)$ is a tree edge, determine the tree path from
$y$ to the $(v_i, w_i)$ cycle by following parent pointers from $y$. Let $z$ be the
vertex on the $(v_i, w_i)$ cycle reached during this search. Compute the total cost
of all vertices except $z$ on this tree path. Scan the tree edges inside the $(y,
w_i)$ cycle, alternately scanning an edge in one cycle and an edge in the other cycle.
Stop scanning when all edges inside one of the cycles have been scanned. Compute the
cost inside this cycle by summing the associated costs of all scanned edges. Use this
cost, the cost inside the $(v_i, w_i)$ cycle, and the cost on the tree path from $y$
to $z$ to compute the cost inside the other cycle. Let $(v_{i+1}, w_{i+1})$ be the
edge among $(v_i, y)$ and $(y, w_i)$ whose cycle has more cost inside it.

Repeat Step~9 until finding a cycle whose inside has cost not exceeding $2/3$.

\begin{center}
\textit{Time:} $\bigO(n)$ (see proof below).
\end{center}

\noindent\textit{Step 10.} Use the cycle found in Step~9 and the levels found in
Step~4 to construct a satisfactory vertex partition as described in the proof of
Lemma~\ref{lem:levels}. Extend this partition from the connected component chosen in
Step~2 to the entire graph as described in the proof of Theorem~\ref{thm:main}.

\begin{center}
\textit{Time:} $\bigO(n)$.
\end{center}

This completes our presentation of the algorithm. All steps except Step~9 obviously
run in $\bigO(n)$ time. We urge readers to fill in the details of this algorithm; we
content ourselves here with proving that Step~9 requires $\bigO(n)$ time.

\begin{proof}[Proof of Step~9 time bound]
Each iteration of Step~9 deletes at least one face from the inside of the current
cycle. Thus Step~9 terminates after $\bigO(n)$ iterations. The total running time of
one iteration of Step~9 is $\bigO(1)$ plus time proportional to the length of the
tree path from $y$ to $z$ plus time proportional to the number of edges scanned
inside the $(v_i, y)$ and $(y, w_i)$ cycles. Each vertex on the tree path from $y$
to $z$ (except $z$) is inside the current cycle but on the boundary or outside of
all subsequent cycles. For every two edges scanned during an iteration of Step~9, at
least one edge is inside the current cycle but outside all subsequent cycles. It
follows that the total time spent traversing tree paths and scanning edges, during all
iterations of Step~9, is $\bigO(n)$. Thus the total time spent in Step~9 is
$\bigO(n)$.
\end{proof}

By making minor modifications to this algorithm, one can construct an $\bigO(n)$ time
algorithm to find a vertex partition satisfying Theorem~\ref{thm:finite-element}, and
$\bigO(n)$ time algorithms to find vertex partitions satisfying
Corollary~\ref{cor:sqrt-sep} and Corollary~\ref{cor:finite-element}.

% ====================================================================
\section{Applications}

The separator theorem proved in \S2 allows us to obtain many new complexity results
since it opens the way for efficient application of divide-and-conquer on planar
graphs. We mention a few such applications here; we shall present the details in a
subsequent paper.

\paragraph{Generalized nested dissection.} Any system of linear equations whose
sparsity structure corresponds to a planar or finite element graph can be solved in
$\bigO(n^{3/2})$ time and $\bigO(n \log n)$ space. This result generalizes the nested
dissection method of George~\cite{george73}.

\paragraph{Pebbling.} Any $n$-vertex planar acyclic directed graph with maximum
in-degree $k$ can be pebbled using $\bigO(\sqrt{n} + k\log n)$ pebbles.
See~\cite{hopcroft77,paul77} for a description of the pebble game.

\paragraph{The post office problem.} Knuth's ``post office'' problem~\cite{knuth73}
can be solved in $\bigO((\log n)^2)$ time and $\bigO(n)$ space. See~\cite{dobkin76,shamos}
for previous results.

\paragraph{Data structure embedding problems.} Any planar data structure can be
efficiently embedded into a balanced binary tree. See~\cite{demillo76,lipton76} for a
description of the problem and some related results.

\paragraph{Lower bounds on Boolean circuits.} Any planar circuit for computing
Boolean convolution contains at least $cn^2$ gates for some positive constant $c$.

% ====================================================================
\section*{Appendix: Graph-theoretic definitions}

A \emph{graph} $G = (V, E)$ consists of a set $V$ of \emph{vertices} and a set $E$
of \emph{edges}. Each edge is an unordered pair $(v, w)$ of distinct vertices. If
$(v, w)$ is an edge, $v$ and $w$ are \emph{adjacent} and $(v, w)$ is \emph{incident}
to both $v$ and $w$. A \emph{path of length} $k$ with \emph{endpoints} $v$, $w$ is a
sequence of vertices $v = v_0, v_1, v_2, \ldots, v_k = w$ such that $(v_{i-1}, v_i)$
is an edge for $1 \leq i \leq k$. If all the vertices $v_0, v_1, \ldots, v_{k-1}$ are
distinct, the path is \emph{simple}. If $v = w$, the path is a \emph{cycle}. The
\emph{distance} from $v$ to $w$ is the length of the shortest path from $v$ to $w$.
(The distance is infinite if $v$ and $w$ are not joined by a path.) The \emph{level}
of a vertex $v$ in a graph $G$ with respect to a fixed root $r$ is the distance from
$r$ to $v$.

If $G_1 = (V_1, E_1)$ and $G_2 = (V_2, E_2)$ are graphs, $G_1$ is a
\emph{subgraph} of $G_2$ if $V_1 \subseteq V_2$ and $E_1 \subseteq E_2$. $G_1$ is a
\emph{generalized subgraph} of $G_2$ if $V_1 \subseteq V_2$ and there is a mapping
$f$ from $E_1$ into the set of paths of $G_2$ such that, for each edge $(v, w) \in
E_1$, $f((v,w))$ has endpoints $v$ and $w$, and no two paths $f((v_1, w_1))$ and
$f((v_2, w_2))$ share a vertex except possibly an endpoint of both paths. If
$G = (V, E)$ is a graph and $V_1 \subseteq V_2$, the graph $G_1 = (V_1, E_1)$ where
$E_1 = E_2 \cap \{(v, w) \mid v, w \in V_1\}$ is the \emph{subgraph of $G_2$ induced
by the vertex set} $V_1$. If $G_1 = (V_1, E_1)$ is a subgraph of $G_2 = (V_2, E_2)$,
then \emph{shrinking $G_1$ to a single vertex in $G_2$} means forming a new graph
$G_2'$ from $G_2$ by deleting from $G_2$ all vertices in $V_1$ and all their incident
edges, adding a new vertex $x$ to $G_2$, and adding a new edge $(x, w)$ to $G_2$ for
each edge $(v, w) \in E_2$ such that $v \in V_1$ and $w \notin V_1$.

A graph is \emph{connected} if any two vertices in it are joined by a path. The
\emph{connected components} of a graph are its maximal connected subgraphs. A
\emph{clique} is a graph such that any two vertices are joined by an edge. A
\emph{tree} is a connected graph containing no cycles. We shall generally assume
that a tree has a distinguished vertex, called a \emph{root}. If $T$ is a tree with
root $r$ and $v$ is on the (unique) simple path from $r$ to $w$, $v$ is an
\emph{ancestor} of $w$ and $w$ is a \emph{descendant} of $v$. If in addition $(v,
w)$ is an edge of $T$, then $v$ is the \emph{parent} of $w$ and $w$ is a
\emph{child} of $v$. The \emph{radius} of a tree is the maximum distance of any
vertex from the root. A \emph{spanning tree} $T$ of a graph $G$ is a subgraph of $G$
which is a tree and which contains all the vertices of $G$. $T$ is a
\emph{breadth-first spanning tree} with respect to a root $r$ if, for any vertex $v$,
the distance from $r$ to $v$ in $T$ is equal to the distance from $r$ to $v$ in $G$.

A graph $G = (V, E)$ is \emph{planar} if there is a one-to-one map $f_1$ from $V$
into points in the plane and a map $f_2$ from $E$ into simple curves in the plane
such that, for each edge $(v, w) \in E$, $f_2((v, w))$ has endpoints $f_1(v)$ and
$f_2(w)$, and no two curves $f_2((v_1, w_1))$, $f_2((v_2, w_2))$ share a point
except possibly a common endpoint. Such a pair of maps $f_1$, $f_2$ is a \emph{planar
embedding} of $G$. The connected planar regions formed when the ranges of $f_1$ and
$f_2$ are deleted from the plane are called the \emph{faces} of the embedding. Each
face is bounded by a curve corresponding to a cycle of $G$, called the
\emph{boundary} of the face. We shall sometimes not distinguish between a face and
its boundary. A \emph{diagonal} of a face is an edge $(v, w)$ such that $v$ and $w$
are nonadjacent vertices on the boundary of the face.

% ====================================================================
\section*{Acknowledgments}

We would like to thank Stanley Eisenstat, Rich A.\ DeMillo, Robert Floyd, Donald
Rose, and Daniel Sleator for many helpful discussions and much thoughtful criticism.

% ====================================================================
\begin{thebibliography}{99}

\bibitem{aho74}
A.~V. Aho, J.~E. Hopcroft, and J.~D. Ullman, \textit{The Design and Analysis of
Computer Algorithms}, Addison-Wesley, Reading, MA, 1974.

\bibitem{demillo76}
R.~A. DeMillo, S.~C. Eisenstat, and R.~J. Lipton, \textit{Preserving average
proximity in arrays}, Georgia Institute of Tech., Tech.\ Rep., Atlanta, 1976.

\bibitem{dobkin76}
D. Dobkin and R.~J. Lipton, \textit{Multi-dimensional searching problems}, SIAM
J.\ Comput., 5 (1976), pp.~181--186.

\bibitem{erdos75}
P. Erd\H{o}s, R.~L. Graham, and E. Szemer\'{e}di, \textit{On sparse graphs with
dense long paths}, STAN-CS-75-504, Computer Sci.\ Dept., Stanford Univ., Stanford,
CA, 1975.

\bibitem{george73}
J.~A. George, \textit{Nested dissection of a regular finite element mesh}, SIAM
J.\ Numer.\ Anal., 10 (1973), pp.~345--367.

\bibitem{hall55}
D.~W. Hall and G. Spencer, \textit{Elementary Topology}, John Wiley, New York, 1955.

\bibitem{harary69}
F. Harary, \textit{Graph Theory}, Addison-Wesley, Reading, MA, 1969.

\bibitem{hopcroft77}
J. Hopcroft, W. Paul, and L. Valiant, \textit{On time versus space}, J.\ Assoc.\
Comput.\ Mach., 24 (1977), pp.~332--337.

\bibitem{hopcroft73}
J.~E. Hopcroft and R.~E. Tarjan, \textit{Efficient algorithms for graph
manipulation}, Comm.\ ACM, 16 (1973), pp.~372--378.

\bibitem{hopcroft74}
J.~E. Hopcroft and R.~E. Tarjan, \textit{Efficient planarity testing}, J.\ Assoc.\
Comput.\ Mach., 21 (1974), pp.~549--568.

\bibitem{knuth73}
D.~E. Knuth, \textit{The Art of Computer Programming, Volume~3: Sorting and
Searching}, Addison-Wesley, Reading, MA, 1973.

\bibitem{kuratowski30}
C. Kuratowski, \textit{Sur le probl\`{e}me des courbes gauches en topologie}, Fund.\
Math., 15 (1930), pp.~271--283.

\bibitem{lewis65}
P.~M. Lewis, R.~E. Stearns, and J. Hartmanis, \textit{Memory bounds for recognition
of context-free and context-sensitive languages}, IEEE Conference on Switching Theory
and Logical Design, IEEE, New York, 1965, pp.~191--202.

\bibitem{lipton76}
R.~J. Lipton, S.~C. Eisenstat, and R.~A. DeMillo, \textit{Space and time
hierarchies for classes of control structures and data structures}, J.\ Assoc.\
Comput.\ Mach., 23 (1976), pp.~710--732.

\bibitem{paterson72}
M.~S. Paterson, \textit{Tape bounds for time-bounded Turing machines}, J.\ Comput.\
System Sci., 6 (1972), pp.~116--124.

\bibitem{paul77}
W.~J. Paul, R.~E. Tarjan, and J.~R. Celoni, \textit{Space bounds for a game on
graphs}, Math.\ Systems Theory, 10 (1977), pp.~239--251.

\bibitem{shamos}
M.~J. Shamos, \textit{Problems in computational geometry}, unpublished manuscript.

\end{thebibliography}

\end{document}
